\documentclass[12pt,a4paper]{article}
\usepackage[UTF8]{ctex}
\usepackage{geometry}
\usepackage{booktabs}
\usepackage{longtable}
\usepackage{array}
\usepackage{graphicx}
\usepackage{xcolor}
\usepackage{colortbl}
\usepackage{fancyhdr}
\usepackage{titlesec}
\usepackage{enumitem}
\usepackage{amsmath}
\usepackage{multirow}

% 页面设置
\geometry{left=2.5cm,right=2.5cm,top=2.5cm,bottom=2.5cm}

% 页眉页脚
\pagestyle{fancy}
\fancyhf{}
\fancyhead[L]{\small 上交所股票型ETF涨跌停持仓分析}
\fancyhead[R]{\small 2026年2月5日}
\fancyfoot[C]{\thepage}

% 标题格式
\titleformat{\section}{\large\bfseries}{\thesection}{1em}{}
\titleformat{\subsection}{\normalsize\bfseries}{\thesubsection}{1em}{}

% 颜色定义
\definecolor{upcolor}{RGB}{220,50,50}
\definecolor{downcolor}{RGB}{50,150,50}
\definecolor{headercolor}{RGB}{240,240,240}
\definecolor{lightgray}{RGB}{250,250,250}

\begin{document}

% 标题页
\begin{titlepage}
\centering
\vspace*{3cm}
{\Huge\bfseries 上交所股票型ETF\\[0.5cm]涨跌停持仓分析报告}\\[2cm]
{\Large 分析日期:2026年2月5日}\\[1cm]
{\large 数据来源:东方财富}\\[3cm]
\rule{\textwidth}{1pt}\\[0.5cm]
{\large\itshape 统计上交所所有股票型ETF收盘后持有涨停股和跌停股的金额}\\[0.5cm]
\rule{\textwidth}{1pt}
\vfill
{\large 云碾网络}\\[0.5cm]
{\normalsize 自动生成报告}
\end{titlepage}

% 目录
\tableofcontents
\newpage

%========================================
\section{报告摘要}
%========================================

本报告统计了上交所股票型ETF在2026年2月5日收盘后持有涨停股和跌停股的情况。通过分析84只ETF的持仓数据,筛选出持有涨停股和跌停股的ETF,并按持仓数量和金额进行排序。

\subsection{核心数据}

\begin{table}[htbp]
\centering
\begin{tabular}{lc}
\toprule
\rowcolor{headercolor}
\textbf{统计项目} & \textbf{数值} \\
\midrule
分析ETF总数 & 84只 \\
当日涨停股票数量 & 44只 \\
当日跌停股票数量 & 15只 \\
\midrule
持有涨停股的ETF数量 & 20只 \\
持有跌停股的ETF数量 & 10只 \\
\midrule
\textcolor{upcolor}{涨停股总持仓金额} & \textcolor{upcolor}{24.89万元} \\
\textcolor{downcolor}{跌停股总持仓金额} & \textcolor{downcolor}{17.57万元} \\
\bottomrule
\end{tabular}
\caption{分析数据汇总}
\end{table}

%========================================
\section{涨停股持仓分析}
%========================================

\subsection{涨停股持仓ETF排名(按数量)}

以下ETF持有涨停股数量最多,投资者需关注相关ETF的流动性风险。

\begin{table}[htbp]
\centering
\small
\begin{tabular}{clccc}
\toprule
\rowcolor{headercolor}
\textbf{排名} & \textbf{ETF名称} & \textbf{代码} & \textbf{涨停股数量} & \textbf{金额(万元)} \\
\midrule
1 & 1000ETF & 512100 & \textcolor{upcolor}{\textbf{10}} & 3.82 \\
2 & 500ETF & 510500 & \textcolor{upcolor}{\textbf{7}} & 7.53 \\
3 & 2000ETF & 560010 & \textcolor{upcolor}{\textbf{4}} & 1.28 \\
4 & 光伏ETF & 515790 & \textcolor{upcolor}{\textbf{3}} & 4.13 \\
5 & 证券ETF & 512880 & \textcolor{upcolor}{\textbf{3}} & 2.46 \\
6 & 煤炭ETF & 515220 & \textcolor{upcolor}{\textbf{3}} & 1.84 \\
7 & 软件ETF & 512660 & \textcolor{upcolor}{\textbf{3}} & 1.54 \\
8 & 芯片龙头 & 512200 & \textcolor{upcolor}{\textbf{3}} & 0.55 \\
9 & 稀土ETF & 515210 & 2 & 0.34 \\
10 & 军工ETF & 512670 & 1 & 0.34 \\
\bottomrule
\end{tabular}
\caption{涨停股持仓ETF排名(前10名)}
\end{table}

\subsection{涨停股明细}

本日被ETF持有的主要涨停股票包括:

\begin{table}[htbp]
\centering
\small
\begin{tabular}{llp{8cm}}
\toprule
\rowcolor{headercolor}
\textbf{股票代码} & \textbf{股票名称} & \textbf{被持有ETF} \\
\midrule
002291 & 遥望科技 & 1000ETF \\
601015 & 陕西黑猫 & 1000ETF、煤炭ETF \\
603103 & 横店影视 & 1000ETF \\
002506 & 协鑫集成 & 1000ETF、光伏ETF、AIETF、锂电ETF、新能源车、养老ETF \\
601187 & 厦门银行 & 1000ETF、银行ETF \\
002985 & 北摩高科 & 1000ETF、军工ETF、软件ETF、影视ETF \\
000636 & 风华高科 & 500ETF \\
002945 & 华林证券 & 500ETF、证券ETF、基建ETF、国企改革 \\
603826 & 坤彩科技 & 500ETF、有色ETF \\
600683 & 京投发展 & 芯片龙头 \\
600382 & 广东明珠 & 稀土ETF \\
601599 & 浙文影业 & 国企ETF \\
\bottomrule
\end{tabular}
\caption{涨停股被持有情况明细}
\end{table}

%========================================
\section{跌停股持仓分析}
%========================================

\subsection{跌停股持仓ETF排名(按数量)}

以下ETF持有跌停股数量最多,可能面临净值下跌压力。

\begin{table}[htbp]
\centering
\small
\begin{tabular}{clccc}
\toprule
\rowcolor{headercolor}
\textbf{排名} & \textbf{ETF名称} & \textbf{代码} & \textbf{跌停股数量} & \textbf{金额(万元)} \\
\midrule
1 & 光伏ETF & 515790 & \textcolor{downcolor}{\textbf{7}} & 4.93 \\
2 & 500ETF & 510500 & \textcolor{downcolor}{\textbf{4}} & 7.42 \\
3 & 1000ETF & 512100 & \textcolor{downcolor}{\textbf{4}} & 2.23 \\
4 & 2000ETF & 560010 & \textcolor{downcolor}{\textbf{3}} & 0.85 \\
5 & 养老ETF & 512580 & \textcolor{downcolor}{\textbf{3}} & 0.18 \\
6 & 稀土ETF & 515210 & 2 & 1.06 \\
7 & AIETF & 516160 & 1 & 0.52 \\
8 & 有色ETF & 512400 & 1 & 0.36 \\
9 & 锂电ETF & 516850 & 1 & 0.01 \\
10 & 新能源车 & 516580 & 1 & 0.01 \\
\bottomrule
\end{tabular}
\caption{跌停股持仓ETF排名(前10名)}
\end{table}

\subsection{跌停股明细}

本日被ETF持有的主要跌停股票包括:

\begin{table}[htbp]
\centering
\small
\begin{tabular}{llp{8cm}}
\toprule
\rowcolor{headercolor}
\textbf{股票代码} & \textbf{股票名称} & \textbf{被持有ETF} \\
\midrule
600481 & 双良节能 & 光伏ETF \\
002518 & 科士达 & 光伏ETF、1000ETF、2000ETF、养老ETF \\
002865 & 钧达股份 & 光伏ETF、养老ETF \\
601212 & 白银有色 & 500ETF、有色ETF \\
601615 & 明阳智能 & 500ETF、AIETF、锂电ETF、新能源车、养老ETF \\
002716 & 湖南白银 & 1000ETF、2000ETF \\
000923 & 河钢资源 & 1000ETF、2000ETF、稀土ETF \\
\bottomrule
\end{tabular}
\caption{跌停股被持有情况明细}
\end{table}

%========================================
\section{重点ETF分析}
%========================================

\subsection{宽基指数ETF}

宽基指数ETF由于持仓分散,涨跌停股影响相对有限。

\begin{table}[htbp]
\centering
\small
\begin{tabular}{lccccc}
\toprule
\rowcolor{headercolor}
\textbf{ETF名称} & \textbf{代码} & \textbf{涨停数} & \textbf{涨停金额} & \textbf{跌停数} & \textbf{跌停金额} \\
\midrule
50ETF & 510050 & 0 & 0.00 & 0 & 0.00 \\
300ETF & 510300 & 0 & 0.00 & 0 & 0.00 \\
500ETF & 510500 & 7 & 7.53 & 4 & 7.42 \\
1000ETF & 512100 & 10 & 3.82 & 4 & 2.23 \\
2000ETF & 560010 & 4 & 1.28 & 3 & 0.85 \\
科创50 & 588000 & 0 & 0.00 & 0 & 0.00 \\
\bottomrule
\end{tabular}
\caption{宽基指数ETF涨跌停持仓(单位:万元)}
\end{table}

\textbf{分析:}小市值指数ETF(如500ETF、1000ETF、2000ETF)由于成分股更容易出现涨跌停,相关持仓较多。而50ETF、300ETF等大盘蓝筹ETF则基本无涨跌停持仓。

\subsection{行业主题ETF}

行业主题ETF持仓集中,受涨跌停影响更为明显。

\begin{table}[htbp]
\centering
\small
\begin{tabular}{lccccc}
\toprule
\rowcolor{headercolor}
\textbf{ETF名称} & \textbf{代码} & \textbf{涨停数} & \textbf{涨停金额} & \textbf{跌停数} & \textbf{跌停金额} \\
\midrule
\textcolor{upcolor}{光伏ETF} & 515790 & 3 & 4.13 & \textcolor{downcolor}{\textbf{7}} & \textcolor{downcolor}{\textbf{4.93}} \\
证券ETF & 512880 & 3 & 2.46 & 0 & 0.00 \\
煤炭ETF & 515220 & 3 & 1.84 & 0 & 0.00 \\
稀土ETF & 515210 & 2 & 0.34 & 2 & 1.06 \\
AIETF & 516160 & 1 & 0.31 & 1 & 0.52 \\
\bottomrule
\end{tabular}
\caption{行业主题ETF涨跌停持仓(单位:万元)}
\end{table}

\textbf{分析:}\textcolor{downcolor}{光伏ETF}本日持有跌停股数量最多(7只),需关注新能源板块的风险。证券ETF和煤炭ETF则表现较好,涨停股持仓较多。

%========================================
\section{投资风险提示}
%========================================

\subsection{涨停股持仓风险}

\begin{itemize}[leftmargin=2em]
\item \textbf{流动性风险:}涨停股无法卖出,ETF面临赎回压力时可能产生折价
\item \textbf{估值风险:}连续涨停可能导致ETF净值与实际价值偏离
\item \textbf{溢价风险:}部分ETF可能因持有涨停股而产生场内溢价
\end{itemize}

\subsection{跌停股持仓风险}

\begin{itemize}[leftmargin=2em]
\item \textbf{净值下跌风险:}跌停股无法及时止损,可能持续拖累ETF净值
\item \textbf{折价风险:}投资者预期继续下跌可能导致ETF场内折价
\item \textbf{行业风险:}跌停股集中的行业ETF(如光伏ETF)需特别关注
\end{itemize}

\subsection{投资建议}

\begin{enumerate}[leftmargin=2em]
\item 关注持有涨停股较多的ETF(如1000ETF、500ETF),警惕溢价风险
\item 规避持有跌停股较多的ETF(如光伏ETF),防范进一步下跌
\item 宽基指数ETF(50ETF、300ETF)受涨跌停影响较小,适合稳健投资
\item 行业ETF波动较大,建议结合行业基本面综合判断
\end{enumerate}

%========================================
\section{附录:数据说明}
%========================================

\subsection{数据来源}
\begin{itemize}[leftmargin=2em]
\item 涨跌停股票数据:东方财富涨停板池/跌停板池
\item ETF持仓数据:东方财富基金持仓明细
\item 数据获取时间:2026年2月5日收盘后
\end{itemize}

\subsection{统计口径}
\begin{itemize}[leftmargin=2em]
\item 涨停判断:涨跌幅达到或超过10\%(ST股票5\%)
\item 跌停判断:涨跌幅达到或超过-10\%(ST股票-5\%)
\item ETF范围:上交所上市的股票型ETF(代码以5开头),排除债券、货币、商品类ETF
\item 持仓金额:基于最新披露的持仓数据计算
\end{itemize}

\subsection{免责声明}

本报告仅供参考,不构成任何投资建议。投资者应根据自身情况独立判断,自行承担投资风险。数据可能存在延迟或误差,请以官方披露为准。

\vspace{1cm}
\begin{center}
\rule{0.5\textwidth}{0.5pt}\\[0.3cm]
{\small 报告生成时间:2026年2月5日}\\
{\small 云碾网络 · 数据分析团队}
\end{center}

\end{document}
